\title{Endowing Coq with Sized Types}
\title{Coq Sized Types Matter}
\title{Sizing Up Coq's Types}
\title{A Bigger, Sized Coq}
\title{SuccCoq: Successor Sized Types in the Calculus of (Co)Inductive Constructions}
\title{The Rise of Coq with Sized Types}
\title{Practical Sized Typing for Coq}

\author{Jonathan Chan}
\orcid{0000-0003-0830-3180}
\affiliation{University of British Columbia}
\email{jon@alumni.ubc.ca}

\author{William J. Bowman}
\affiliation{University of British Columbia}
\email{wilbowma@cs.ubc.ca}

\usepackage{amsmath}
\usepackage{amsfonts}
\usepackage{mathtools}
\usepackage{stmaryrd}
\usepackage{minted}
\usepackage{bussproofs}
\usepackage{cuted}
\usepackage{hyphenat}
\usepackage{stackengine}
\usepackage{xspace}
\usepackage[shortcuts]{extdash}
\usepackage{soul}
\usepackage{enumitem}
\usepackage{balance}
\usepackage[title]{appendix}

% Change list indent to 1 em so that bullets are flush with paragraph indents
\setlist{leftmargin=1em}
% Use \autoref for references with prefixes
\def\sectionautorefname{Section}
\def\subsectionautorefname{Subsection}
\def\subsubsectionautorefname{Subsubsection}
\def\figureautorefname{Figure}
\def\tableautorefname{Table}
\def\Appendixautorefname{Appendix}
% Coloured reference links
\hypersetup{colorlinks = true}

% rule hyperlinks
\newcommand{\defrule}[1]{\phantomsection\label{#1}(#1)}
\newcommand{\defnamerule}[2]{\phantomsection\label{#1}(#2)}
\newcommand{\refrule}[1]{\hyperref[#1]{Rule (#1)}}
\newcommand{\refnorule}[1]{\hyperref[#1]{(#1)}}
% langs
\newcommand{\lang}{CIC$\widehat{*}$\xspace}
\newcommand{\CIChat}{CIC$\widehat{~}$\xspace}
\newcommand{\CIChatbar}{CIC$\widehat{\text{\underline{\:\:}}}$\xspace}
\newcommand{\Fhat}{F$\widehat{~}$\xspace}
\newcommand{\CIChatl}{CIC$\widehat{_l}$\xspace}
\newcommand{\CChatomega}{CC$\widehat{\omega}$\xspace}
\newcommand{\CIChatsub}{CIC$\widehat{_\sqsubseteq}$\xspace}
% metafunctions (for math envs)
\newcommand{\dom}[1]{\text{dom}(#1)}
\newcommand{\SV}[1]{\text{SV}(#1)}
\newcommand{\WF}[1]{\text{WF}(#1)}
\newcommand{\whnf}[1]{\textsc{whnf}(#1)}
\newcommand{\fresh}[1]{\textsc{fresh}(#1)}
\newcommand{\freshi}[1]{\textsc{fresh*}(#1)}
\newcommand{\axiom}[1]{\textsc{axiom}(#1)}
\newcommand{\rules}[2]{\textsc{rule}(#1, #2)}
\newcommand{\elim}[3]{\textsc{elim}(#1, #2, #3)}
\newcommand{\inds}[1]{\textsc{inds}(#1)}
\newcommand{\indtype}[2]{\textsc{indType}(#1, #2)}
\newcommand{\constrtype}[3]{\textsc{constrType}(#1, #2, #3)}
\newcommand{\motivetype}[4]{\textsc{motiveType}(#1, #2, #3, #4)}
\newcommand{\branchtype}[5]{\textsc{branchType}(#1, #2, #3, #4, #5)}
\newcommand{\indsig}[1]{\textsc{indSig}(#1)}
\newcommand{\constrsig}[1]{\textsc{constrSig}(#1)}
\newcommand{\decompose}[2]{\textsc{decompose}(#1, #2)}
\newcommand{\shift}[1]{\textsc{shift}(#1)}
\newcommand{\setrecstars}[2]{\textsc{setRecStars}(#1, #2)}
\newcommand{\setcorecstars}[1]{\textsc{setCorecStars}(#1)}
\newcommand{\getrecvar}[2]{\textsc{getRecVar}(#1, #2)}
\newcommand{\getcorecvar}[1]{\textsc{getCorecVar}(#1)}
\newcommand{\getposvars}[2]{\textsc{getPosVars}(#1, #2)}
\newcommand{\RecCheckLoop}[5]{\textsc{RecCheckLoop}(#1, #2, #3, #4, #5)}
% terms (for math envs)
\newcommand{\Prop}{\text{Prop}}
\newcommand{\Set}{\text{Set}}
\newcommand{\Type}[1]{\text{Type}_{#1}}
\newcommand{\letin}[4]{\text{let}\ #1 : #2 \coloneqq #3\ \text{in}\ #4}
\newcommand{\caseof}[4]{\text{case}_{#1}\ #2\ \text{of}\ \langle #3 \Rightarrow #4 \rangle}
\newcommand{\fix}[4]{\text{fix}_{#1}\ \langle #2 : #3 \coloneqq #4 \rangle}
\newcommand{\cofix}[4]{\text{cofix}_{#1}\ \langle #2 : #3 \coloneqq #4 \rangle}
% other (for math envs)
\newcommand{\set}[1]{\{#1\}}
\newcommand{\sgg}{\Sigma, \Gamma_G, \Gamma}
\newcommand{\cgg}{C, \Gamma_G, \Gamma}
\newcommand{\pos}{\ \text{pos}\ }
\renewcommand{\neg}{\ \text{neg}\ }
% other
\newcommand{\spacer}{\vspace{1ex}}
% hyphenation
\newcommand{\coinductive}{(co)\-inductive\xspace}
\newcommand{\corecursive}{(co)\-recursive\xspace}
\newcommand{\cofixpoint}{(co)\-fix\-point\xspace}
\newcommand{\cofixpoints}{(co)\-fix\-points\xspace}

\hyphenation{para-metrized}
\hyphenation{re-duc-tion}